\documentclass[12pt]{constitution}
\usepackage{mathpazo}
\usepackage[francais]{babel}
\usepackage[T1]{fontenc}
\usepackage[utf8]{inputenc}
\usepackage{array, multirow, tabularx}
\usepackage{graphicx}
% TODO : Adapter la classe constitution.cls pour y inclure ces définitions modifiées
\titleformat{\article}[display]{\filcenter\normalfont\bfseries}{Titre \thearticle}{0pt}{\filcenter\normalfont\bfseries}
\titleformat{\section}[runin]{\normalfont\scshape}{Article \thesection.}{.5em}{}[\quad]
\begin{document}
\title{<Nom de l'association> - Statuts}
\author{Association déclarée par application de la loi du 1er juillet 1901 et du décret du 16 août 1901}
\date{<Date>}
\maketitle
\newpage
	
\article{Présentation de l'association}
\section{Nom}
Il est fondé entre les adhérents aux présents statuts une association régie par la loi du 1er juillet 1901 et le décret du 16 août 1901, ayant pour titre <Nom de l'association>.

\section{Objet}
Cette association a pour objet de :
\begin{itemize}
	\item <Premier objet de l'association>
	%\item <Deuxième objet de l'association>
\end{itemize}

\section{Siège social}
Le siège social est fixé à <Adresse complète du siège>.

Il pourra être transféré par simple décision du conseil d'administration; la ratification par l’assemblée générale sera nécessaire.

\section{Durée}
La durée de l’association est illimitée.

\article{Composition de l'association}
% TODO : Adapter la classe constitution.cls pour ne pas remettre à 0 le compteur d'articles à chaque titre
\setcounter{section}{4}

\section{Membres}
L'association se compose de :
\begin{enumerate}
	\item Membres actifs;
	\item Membres d'honneur;
	\item Membres bienfaiteurs.
\end{enumerate}

\medskip
Les membres actifs, personnes physiques ou morales, acquittent une cotisation fixée annuellement par l'Assemblée Générale. Ils sont membres de l'Assemblée Générale avec voix délibérative.

\medskip
Les membres d'honneur sont désignés par l'Assemblée Générale pour les services qu'ils ont rendus ou rendent à l'association. Ils sont dispensés du paiement de la cotisation annuelle et ont le droit de participer à l'Assemblée Générale sans voix délibérative.

\medskip
Les membres bienfaiteurs qui acquittent une cotisation spéciale fixée par l'Assemblée Générale ont le droit de participer à l'Assemblée Générale avec voix délibérative.

\section{Admission}
Pour faire partie de l'association, il faut être agréé par le conseil d’administration, qui statue, lors de chacune de ses réunions, sur les demandes d'admission présentées.

\section{Radiation}
La qualité de membre se perd par :
\begin{enumerate}
	\item La démission;
	\item Le décès;
	\item La radiation prononcée par le conseil d'administration pour non-paiement de la cotisation ou pour motif grave, l'intéressé ayant été invité par lettre recommandée à fournir des explications devant le conseil d'administration et/ou par écrit. 
\end{enumerate}

\end{document}