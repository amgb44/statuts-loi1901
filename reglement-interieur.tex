\documentclass[a4paper,twoside]{article}

\usepackage[english,francais]{babel}
\usepackage[latin1]{inputenc}
\usepackage{eurosym}
\usepackage[hmargin=3cm,vmargin=3.5cm]{geometry}
\usepackage{ntheorem}
\usepackage{url}

\let\urlorig\url
\renewcommand{\url}[1]{%
  \begin{otherlanguage}{english}\urlorig{#1}\end{otherlanguage}%
}

\theoremstyle{plain} 
\theoremheaderfont{\scshape} 
\theorembodyfont{\large\bfseries}
\theoremseparator{~:~}
\newtheorem{Art}{Article}


\title{Règlement intérieur de l'association TuxFamily.org}

\author{TuxFamily.org}

\begin{document}

Règlement intérieur de TuxFamily.org, association à but non lucratif régie par la loi
du 1er juillet 1901 et le décret du 16 août 1901.

\begin{Art}Préambule\end{Art} 

Ceci   constitue  le   règlement  intérieur   de   l'association  TuxFamily.org.
Conformément aux statuts de l'association, il  est établi et voté par le conseil
d'administration, puis ratifié par  l'assemblée générale. Ce règlement intérieur
a été  ratifié lors  de l'assemblée  générale du 8  juillet 2009.  Tout adhérent
s'engage au vu des statuts et du règlement intérieur de l'association.

\begin{Art}Cotisation\end{Art}

Le montant de la cotisation annuelle est fixé comme suit~:\\

\begin{tabular}{|c|c|}
\hline
Type & Montant\\
\hline
Personne physique & 15 \euro\\
Étudiant, chômeur & 10 \euro\\
Personne morale & 50 \euro\\
\hline
\end{tabular}
~\\

\begin{Art}Communication officielle interne\end{Art}

Sauf  mention  contraire dans  les  statuts  ou  le règlement  (en  particulier,
adhésion et  démission), les communications se  font par voie  numérique dès que
possible,  sauf   demande  contraire,  motivée,  de   l'adhérent  concernant  la
communication lui étant destinée. Les adhérents choisissent librement d'utiliser
la voie  numérique ou la  voie postale pour  leur communication avec  le conseil
d'administration ou un de ses membres.

\begin{Art}Traitement des informations sur les membres\end{Art}

La liste des membres est  publique pour l'ensemble des membres de l'association,
ainsi que pour  les candidats en période d'élection, comme  précisé dans la loi.
En  revanche, l'association  n'est  pas autorisée  à  publier, les  informations
personnelles du membre sans son accord,  i.e. toute information en dehors de ses
noms et prénoms.\\

Lors de l'adhésion, l'association  demande les informations suivantes concernant
l'adhérent~:

\begin{itemize}
\item nom et prénoms,
\item adresse postale complète et tout autre moyen de communication permettant
      de la joindre (la possession d'une adresse email est fortement
      recommandée),
\item téléphone,
\item profession (ou activité)
\end{itemize}
~\\ 

Ces informations font l'objet d'un  traitement informatique et sont destinées au
secrétariat l'association. En application de l'article 34 de la loi du 6 janvier
1978,  les  membres bénéficient  d'un  droit  d'accès  et de  rectification  aux
informations qui les concernent auprès du secrétaire de l'association.\\

L'adhérent  s'engage  à  porter   à  la  connaissance  de  l'association  toutes
modifications   portant   sur  son   adresse   postale,  adresse   électronique,
téléphone.

\begin{Art}Assemblée générale\end{Art}

L'assemblée générale, ordinaire ou extraordinaire, se déroule, de préfèrence, en
présence  physique des  adhérents. Le  conseil d'administration  peut toutefois,
lorsque  cela   est  nécessaire  pour  assurer  la   présence  d'une  proportion
conséquente des adhérents, décider de  tenir une assemblée générale sur un canal
IRC privé et réservé à ses membres.\\

Les  rapports moraux  et financiers  sont mis  à disposition  des  adhérents une
semaine avant l'assemblée générale.  Un compte-rendu de l'assemblée générale est
envoyé aux adhérents quinze jours après l'assemblée générale.\\

Les membres qui ne pourraient se rendre à l'assemblée générale peuvent s'y faire
représenter.  Ils doivent  pour cela  transmettre un  pouvoir signé  à  un autre
membre,  et indiquer  au secrétaire  leur souhait  de se  faire  représenter. Le
nombre de pouvoirs par membre est limité à deux.\\

Les votes s'effectuent à main levée. Toutefois, si un membre au moins en fait la
demande, le vote a lieu à  bulletin secret. Le dépouillement est effectué par le
secrétaire sortant, sous contrôle du président sortant.

\begin{Art}Déclaration de candidature\end{Art}

Les candidatures  au conseil d'administration doivent être  adressées au conseil
d'administration  sortant, au  moins sept  jours  avant la  date de  l'assemblée
générale. Celles-ci devront être accompagnées d'une présentation du candidat, et
éventuellement d'une profession de foi.

\begin{Art}Remboursement des dépenses\end{Art}

En  ce  qui  concerne  les   frais  de  déplacement,  ceux-ci  auront  lieu  sur
présentation du titre de transport ou de la facture des frais engagés (carburant
et péages). Les membres sollicitant le remboursement de leurs frais de carburant
s'engagent  à faire  en sorte  que cette  facture soit  représentative  de leurs
dépenses effectives,  par exemple en faisant  le plein au  début et à la  fin du
voyage et en fournissant la seconde facture.

\begin{Art}Statut des administrateurs\end{Art}

Est  administrateur toute  personne  qui administre  le  matériel appartenant  à
l'association  et dispose,  à cet  effet,  des droits  d'administrateur sur  les
ordinateurs de l'association.  Ils sont cooptés par les  administrateurs déjà en
place.\\

Les  administrateurs doivent  être membres  de l'association.  En cas  de litige
entre  eux,  les  administrateurs  peuvent demander  l'intervention  du  conseil
d'administration.  Ce dernier  a  de toute  facon  un droit  de  regard sur  les
activités  des  administrateurs  et  peut   exiger  à  tout  moment  un  rapport
d'activités de ces derniers.\\

La radiation  d'un administrateur  peut se faire  dans les mêmes  conditions que
celle d'un membre.
\end{document}
